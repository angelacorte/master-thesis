%! Author = angela
%! Date = 24/01/24
% !TeX root = ../thesis-main.tex

%----------------------------------------------------------------------------------------
\chapter{Introduction}
\label{chap:introduction}
%----------------------------------------------------------------------------------------
\section{Context}
\label{sec:context}

\paragraph{Macroprogramming}


\paragraph{Self-Organisation}

\subsection{Aggregate Computing}
\label{subsec:aggregate-computing}
Aggregate Computing is a paradigm and engineering approach for compositional development of self-adaptive \ac{iot} services from a global perspective.

A given environment can be seen as a whole programmable entity whose parts collaboratively produce and consume services
across space and time.

\ac{ac} is based on the manipulation of distributed data structures, called \textit{Computational Fields},
with the aim of conceptually mapping each device to a value produced in a program.

Therefore, its structure supports the specification, analysis, simulation and runtime execution of \textit{collective}
or \textit{aggregate} services, independently of the specific~\ac{iot} architecture.

It has been developed with the core idea of functionality composing collective behaviours to achieve effective and resilient
complex behaviours in dynamic networks.

This paradigm has three key traits that characterise it:
    i) \textit{global stance with global-to-local mapping}, where the target of system design is the entire distributed
        \ac{iot} ecosystem,
    ii) \textit{behaviour compositionality}, whereby a rich collective service can be described in terms of the functional
        composition of simpler collective services, and
    iii) \textit{abstraction}, by which aggregate services enable adaptivity at different levels by abstracting from low-level
        issues and details.

%These characteristics play an essential role from the design perspective, where %

The main goal of \ac{ac} is to evaluate a single program that runs on multiple devices of different nature,

\paragraph{Alignment}

\subsection{Field Calculus}
\label{subsec:field-calculus}

\subsection{XC}
\label{subsec:xc}


\section{Motivations}
\label{sec:motivations}

\subsection{Heterogeneity limitations}
\label{subsec:heterogeneity-limitations}

\subsection{Goal}
\label{subsec:goal}


\section{State of Art}
\label{sec:state-of-art}

\subsection{Protelis}
\label{subsec:protelis}

\subsection{ScaFi}
\label{subsec:scafi}

\subsection{FCPP}
\label{subsec:fcpp}


%
%Write your intro here.
%\sidenote{Add sidenotes in this way. They are named after the author of the thesis}
%
%You can use acronyms that your defined previously,
%such as \ac{IoT}.
%%
%If you use acronyms twice,
%they will be written in full only once
%(indeed, you can mention the \ac{IoT} now without it being fully explained).
%%
%In some cases, you may need a plural form of the acronym.
%%
%For instance,
%that you are discussing \acp{vm},
%you may need both \ac{vm} and \acp{vm}.
%
%\paragraph{Structure of the Thesis}
%
%\note{At the end, describe the structure of the paper}
%
