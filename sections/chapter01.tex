%! Author = angela
%! Date = 24/01/24
% !TeX root = ../thesis-main.tex

%----------------------------------------------------------------------------------------
\chapter{Introduction}
\label{ch:introduction}
%----------------------------------------------------------------------------------------
\section{Context}
\label{sec:context}

Computing devices are becoming cheaper and more \emph{ubiquitous}, increasing the complexity of distributed systems.
It is now common for individuals to own multiple computing devices of varying types,
resulting in technology becoming more integrated into daily activities.
Hence, the need to engineer and coordinate the operations in such systems, one way is to program and operate in terms of
    \empth{aggregates} devices, rather than manage each single device.
In actual fact, the coordination of macroscopic behavior in collective systems through a single program is a form of
~\empth{macroprogramming}.
However, this approach presents some primary challenges such as ensuring resilience, efficiency and privacy.

\paragraph{Macroprogramming}
%Macroprogramming: Concepts, State of the Art, and Opportunities of Macroscopic Behaviour Modelling

\paragraph{Self-Organisation}

\subsection{Aggregate Computing}
\label{subsec:aggregate-computing}

%TODO cite Modelling and simulation of Opportunistic IoT Services with Aggregate Computing and
% From distributed coordination to field calculus and aggregate computing

~\ac{ac} is a paradigm and engineering approach for the compositional development of self-adaptive
~\ac{iot} services from a global perspective.
It has been developed with the core idea of functionality
composing collective behaviours to achieve effective and resilient
complex behaviours in dynamic networks.
It views a given environment as a whole programmable entity whose parts collaboratively produce and consume services
across space and time.
~\ac{ac} is based on the principles of ~\empth{Field Calculus} that is a functional programming model used to specify
and compose collective behaviours with formally equivalent local and aggregate semantics.
The concept of ~\empth{Computational Fields} can be viewed as a distributed data structure,
with the aim of conceptually mapping each device to a value produced in a program, considering both
space and time.
Therefore, its structure supports the specification, analysis, simulation and runtime execution of \empth{collective}
or ~\empth{aggregate} services, independently of the specific~\ac{iot} architecture.

This paradigm has three key traits that characterize it:
    i) \emph{Global stance with global-to-local mapping}, where the target of system design is the entire distributed
        \ac{iot} ecosystem,
    ii) \emph{Behaviour compositionality}, whereby a rich collective service can be described in terms of the functional
        composition of simpler collective services, and
    iii) \emph{Abstraction}, by which aggregate services enable adaptivity at different levels by abstracting from low-level
        issues and details.

%These characteristics play an essential role from the design perspective, where %

The main goal of ~\ac{ac} is to evaluate a single program that runs on multiple devices of different nature,

\paragraph{Alignment}

\subsection{Field Calculus}
\label{subsec:field-calculus}

\subsection{XC}
\label{subsec:xc}


\section{Motivations}
\label{sec:motivations}

\subsection{Heterogeneity limitations}
\label{subsec:heterogeneity-limitations}

\subsection{Goal}
\label{subsec:goal}


\section{State of Art}
\label{sec:state-of-art}

\subsection{Protelis}
\label{subsec:protelis}

\subsection{ScaFi}
\label{subsec:scafi}

\subsection{FCPP}
\label{subsec:fcpp}


%
%Write your intro here.
%\sidenote{Add sidenotes in this way. They are named after the author of the thesis}
%
%You can use acronyms that your defined previously,
%such as \ac{IoT}.
%%
%If you use acronyms twice,
%they will be written in full only once
%(indeed, you can mention the \ac{IoT} now without it being fully explained).
%%
%In some cases, you may need a plural form of the acronym.
%%
%For instance,
%that you are discussing \acp{vm},
%you may need both \ac{vm} and \acp{vm}.
%
%\paragraph{Structure of the Thesis}
%
%\note{At the end, describe the structure of the paper}
%
