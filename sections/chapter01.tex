%! Author = angela
%! Date = 24/01/24
% !TeX root = ../thesis-main.tex

%----------------------------------------------------------------------------------------
\chapter{Introduction}
\label{ch:introduction}
%----------------------------------------------------------------------------------------
\section{Context}
\label{sec:context}

Computing devices are becoming cheaper and more \emph{ubiquitous}, increasing the complexity of distributed systems.
It is now common for individuals to own multiple computing devices of varying types,
resulting in technology becoming more integrated into daily activities.
Hence, the need to engineer and coordinate the operations in such systems, one way is to program and operate in terms of
    \empth{aggregates} devices, rather than manage each single device.
In actual fact, the coordination of macroscopic behavior in collective systems through a single program is a form of
~\empth{macroprogramming}.
However, this approach presents some primary challenges such as ensuring resilience, efficiency and privacy.

\paragraph{Macroprogramming}
%Macroprogramming: Concepts, State of the Art, and Opportunities of Macroscopic Behaviour Modelling
%todo

\paragraph{Self-Organisation}
Coordination modes are based on the notion that interactions among multiple independent and autonomous software systems
can be designed as a space orthogonal to pure computation.
This idea can be reified into a concept of shared data space, enabling so-called \emph{generative communication}.
%Todo talk about Linda and Mars ???
Over the course of time, different approaches have been created, such as \emph{Linda} and \emph{Mars},
suggesting innovative techniques for programming systems with devices of different nature focusing on the coordination
of centralised local components, but not on the distribution of the systems.
The main problems that can be encountered in distributed systems are dealing with
    i) openness, as unexpected environment changes,
    ii) large scale of agents and coordination abstractions to be managed,
    iii) intrinsic adaptiveness, such as the ability to intercept relevant events and react to them, guaranteeing
        the resilience of the system.
The challenges necessitate a \emph{self-organising coordination} approach, wherein coordination abstractions solely
manage logical interactions.
This ensures the emergence of global and robust patterns of correct coordination behaviour.


\subsection{Computational Fields}
\label{subsec:computational-fields}

\paragraph{Field-Based Coordination}
To facilitate self-organisation patterns of agents in complex environments, the concept of \emph{coordination field}
has been introduced.
This abstraction serves as a navigational tool for agents over the actual environment.
%todo cite tota
In this context, the tuple-based middleware TOTA (Tuples On The Air) has been suggested to support field-based
coordination for pervasive-computing applications.
Initially, each field of the tuple in the system was assigned a name, along with a formula that supports the
if-then-else construct and includes arithmetic and boolean operators to specify the field's behaviour over time.
Secondly, it was introduced an operator in the tuple space responsible for applying formulas using contextual information.\\
Somewhat independently of the challenge of identifying appropriate coordination models for distributed and situated systems,
several studies have tackled analogous issues in the broader endeavour of constructing distributed intelligent systems.
This involves promoting higher abstractions of spatial collective adaptive systems.

\paragraph{Field Calculus}
Among the studies such as for managing space-time computing models for the manipulation of distributed data structures,
the notion of \emph{computational fields} were proposed.
Consequently, the ~\ac{fc} has been proposed as a foundational model for the coordination of computational
devices spread in physical environments, also known as \emph{aggregate computing}.
%todo is it necessary to insert the abstract syntax of the field calculus?
~\ac{fc} was introduced as minimal core calculus with the aim of capturing the fundaments that make use of computational
fields: functions over and with fields, their evolution over time and the construction of field of values from neighbours.

The main concept of \ac{fc} is to specify the aggregate system behaviour of a network of devices, where devices that can
directly communicate with each other are indicated through a dynamic network relation.
An example of its application is within a sensor network with a range of a broadcast communication.
The behaviour is applied through a functional composition of operators that manipulate the computational fields,
called specification, that can be interpreted locally or globally.
A local specification can describe a computation on an individual device executed in asynchronous ``computation rounds'',
including sending or receiving messages from neighbours, getting information from sensors and computing the local value of the field.
In the global view, a field calculus expression specifies a mapping associating each computation round of each device to
the value that it assumes at that space-time event.
This dual nature inherently facilitates the alignment of individual device behaviour with the overarching global behaviour
of the entire network of devices.

\subsection{Aggregate Computing}
\label{subsec:aggregate-computing}

%TODO cite Modelling and simulation of Opportunistic IoT Services with Aggregate Computing and
% From distributed coordination to field calculus and aggregate computing

~\ac{ac} is a paradigm and engineering approach for the compositional development of self-adaptive
~\ac{iot} services from a global perspective.
It has been developed with the core idea of functionality
composing collective behaviours to achieve effective and resilient
complex behaviours in dynamic networks.
It views a given environment as a whole programmable entity whose parts collaboratively produce and consume services
across space and time.
~\ac{ac} is based on the principles of ~\empth{Field Calculus} that is a functional programming model used to specify
and compose collective behaviours with formally equivalent local and aggregate semantics.
The concept of ~\empth{Computational Fields} can be viewed as a distributed data structure,
with the aim of conceptually mapping each device to a value produced in a program, considering both
space and time.
Therefore, its structure supports the specification, analysis, simulation and runtime execution of \empth{collective}
or ~\empth{aggregate} services, independently of the specific~\ac{iot} architecture.

This paradigm has three key traits that characterise it:
    i) \emph{Global stance with global-to-local mapping}, where the target of system design is the entire distributed
        \ac{iot} ecosystem,
    ii) \emph{Behaviour compositionality}, whereby a rich collective service can be described in terms of the functional
        composition of simpler collective services, and
    iii) \emph{Abstraction}, by which aggregate services enable adaptivity at different levels by abstracting from low-level
        issues and details.

%These characteristics play an essential role from the design perspective, where %

The main goal of ~\ac{ac} is to evaluate a single program that runs on multiple devices of different nature,

\paragraph{Alignment}

\subsection{XC}
\label{subsec:xc}


\section{Motivations}
\label{sec:motivations}

\subsection{Heterogeneity limitations}
\label{subsec:heterogeneity-limitations}

\subsection{Goal}
\label{subsec:goal}


\section{State of Art}
\label{sec:state-of-art}

\subsection{Protelis}
\label{subsec:protelis}

\subsection{ScaFi}
\label{subsec:scafi}

\subsection{FCPP}
\label{subsec:fcpp}


%
%Write your intro here.
%\sidenote{Add sidenotes in this way. They are named after the author of the thesis}
%
%You can use acronyms that your defined previously,
%such as \ac{IoT}.
%%
%If you use acronyms twice,
%they will be written in full only once
%(indeed, you can mention the \ac{IoT} now without it being fully explained).
%%
%In some cases, you may need a plural form of the acronym.
%%
%For instance,
%that you are discussing \acp{vm},
%you may need both \ac{vm} and \acp{vm}.
%
%\paragraph{Structure of the Thesis}
%
%\note{At the end, describe the structure of the paper}
%
