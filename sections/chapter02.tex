%! Author = angela
%! Date = 24/01/24
% !TeX root = ../thesis-main.tex

\chapter{Contributions}
\label{ch:contributions}
In this chapter, will be presented the main contributions of this thesis.

\section{Collektive}
\label{sec:collektive}

\emph{Collektive} is a framework designed to simplify the definition of \ac{ac} systems.

The main objective of this technology is to facilitate the development of aggregate programs that can be executed on a
variety of computing systems, such as mobile and wearable devices, computers, and the cloud.
This allows for interoperability and communication between these systems, despite their different nature.

To achieve this, \emph{Collektive} uses the \ac{fc} model to provide a straightforward and intuitive method for defining
an aggregate program, without the need for low-level coding.

\section{XC in Collektive}
\label{sec:exchange-in-collektive}

Thanks to the design of \textbf{XC}, it is possible to implement the methods proposed by \emph{field calculus}
(~\ref{par:syntax-of-field-calculus}) in terms of \emph{exchange} (~\ref{par:communication-in-xc}).

The syntax of \emph{XC} allows for sending messages to specific nodes, enabling the implementation of \emph{field calculus}
operations through message exchange.

The \emph{exchange} communication is based on \emph{anisotropic} communications, meaning that it has not the same properties
or characteristics in all directions (\Cref{fig:anisotropic}); therefore, messages have custom values sent to different neighbours.

\begin{figure}[h!]
    \centering
    \includegraphics[width=0.2\textwidth]{figures/anisotropic}
    \caption{Anisotropic communication.}
    \label{fig:anisotropic}
\end{figure}

This concept can be extended to the \emph{share} function of field calculus, with the difference that the
operation of \emph{share} is based on \emph{isotropic} communication, meaning that properties are uniform in all directions
(\Cref{fig:isotropic}); therefore, messages have the same value sent to all neighbours.

\begin{figure}[h!]
    \centering
    \includegraphics[width=0.2\textwidth]{figures/isotropic}
    \caption{Isotropic communication.}
    \label{fig:isotropic}
\end{figure}

\section{DSL}
\label{sec:dsl}

%message modelling (in scafi there is an action that computes the messages and then creates a reaction that
%sends the message) we do it in a different way

\paragraph{Architecture}

\paragraph{Messages}

\paragraph{Syntax}
%code
%rep non serve farla con xc perche non vogliamo che gli altri nodi ricevano il messaggio, cose che non gli interessano

\section{Plugin Extensions}
\label{sec:plugin-extensions}

\paragraph{Alignment}
%how

\section{Incarnation}
\label{sec:incarnation}
%how, why

%gradle task

\section{Technologies}
\label{sec:technologies}
%multiplatform

\section{Implementation}
\label{sec:implementation}
%diagrams



