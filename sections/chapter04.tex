%! Author = angela
%! Date = 24/01/24
% !TeX root = ../thesis-main.tex

\chapter{Conclusions}
\label{ch:conclusions}

The primary objective of this thesis was to extend \emph{Collektive} -- a \ac{dsl} for aggregate computing --
by applying the principles of \xc{} to enhance usability and offer a more expressive and flexible
language for defining complex systems.
For this purpose, it was necessary to explore and deepen the concepts of aggregate computing and \xc{},
in such a way as to identify a solution able to guarantee the correctness of the language and its features,
maintaining a good result in terms of performance.

First, an expanded version of the \ac{dsl} was implemented, changing the syntax and the semantics of the language
applying the principles of \xc{}.
By changing the semantics, it has been necessary to expand the alignment mechanism, to ensure the correct functioning
of the language and its features.
This was done maintaining the goal of having a multiplatform implementation with \ac{kmp}, allowing the execution of the
aggregate programs on different platforms.

Furthermore, an implementation of an incarnation for the \emph{Alchemist} simulator has been provided,
enabling the execution of aggregate programs, using the \emph{Collektive} \ac{dsl}, within its environment.

Finally, benchmarks have been implemented to compare the actual state of the art for aggregate computing
-- \emph{ScaFi} and \emph{Protelis} -- and \emph{Collektive}.
The results showed that the combination of the \ac{dsl} and incarnation developed in this thesis has an overall good
performance.

\section{Future Works}
\label{sec:future-works}
In the near future, the development of the \ac{dsl} will continue, as it has been possible to win a research grant from the GARR organization.
It will therefore be possible to further deepen the research and develop new features for the language.

As future works have been identified the followings:
\begin{itemize}
    \item \textbf{Further optimisations of the \ac{dsl}}: the \ac{dsl} will be further optimised to improve its performance;
    \item \textbf{Alignment optimisation}: to improve performances and security looking forward to the execution of the aggregate
        program on different platforms and environments;
    \item \textbf{Development of a standard library}: to provide modules and functionalities to simplify the writing of the aggregate program.
        Such as self-stabilizing functions that could encompass a range of strategies commonly employed to attain adaptable
        and resilient decentralised behaviours; as the ones proposed by the \emph{Protelis-Lang} library \cite{8064092},
        a \emph{Protelis}' \ac{api} for resilient system design.
    \item \textbf{Creation of demonstrations}: to illustrate the possibility of running the same aggregate program on
        different platforms simultaneously, including server JVM, Android devices, iOS, and web browsers.
\end{itemize}
