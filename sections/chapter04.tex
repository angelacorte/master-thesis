%! Author = angela
%! Date = 24/01/24
% !TeX root = ../thesis-main.tex

\chapter{Conclusions}
\label{ch:conclusions}

The primary objective of this thesis was to enhance the current \ac{dsl} for aggregate computing \emph{Collektive},
leveraging the principles of ~\xc~ to enhance the performance of the \ac{dsl}.
The other main goal was to develop a multi-platform implementation of the \ac{dsl} with \ac{kmp}, to allow the execution
of the aggregate programs on different platforms.

This enhancement was pursued to ensure its competitiveness against existing solutions and to enhance its usability.
Additionally, there was as a goal to develop a corresponding incarnation for conducting simulations within the \emph{Alchemist} environment.

A key aspect in the development of the \ac{dsl} was to ensure a good and functional alignment, to guarantee
the correct functioning of the language and its features, maintaining a good result in terms of performance.
The communication between the nodes, intended as the exchange of messages, was also a fundamental aspect to guarantee
the correct functioning of the language and its execution speed.
All this was possible thanks to the implementation of an expressive and flexible language, which made it easy to define complex systems.

\section{Future Works}
\label{sec:future-works}
In the near future, the development of the \ac{dsl} will continue, as it has been possible to win a research grant from the GARR organization.
It will therefore be possible to further deepen the research and develop new features for the language.

As future works have been identified the followings:
\begin{itemize}
    \item \textbf{Further optimisations of the \ac{dsl}}: the \ac{dsl} will be further optimised to improve its performance;
    \item \textbf{Alignment optimisation}: to improve performances and security looking forward to the execution of the aggregate
        program on different platforms and environments;
    \item \textbf{Development of a standard library}: to provide modules and functionalities to simplify the writing of the aggregate program.
        such as self-stabilizing functions that could encompass a range of strategies commonly employed to attain adaptable
        and resilient decentralised behaviours.
        These functions aim to conceal complexities by leveraging constructs from field calculus;
    \item \textbf{Creation of demonstrations}: to illustrate the possibility of running the same aggregate program on
        different platforms simultaneously, including server JVM, Android devices, iOS, and web browsers.
\end{itemize}
