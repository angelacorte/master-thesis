%! Author = angela
%! Date = 24/01/24
% !TeX root = ../thesis-main.tex

\chapter{Validation}
\label{ch:validation}
Validating the hypotheses is a fundamental step for the correct evaluation of the work done.
By comparing the performance of the new implementation with the original one, it is possible to understand if the
introduction of the new features has led to an improvement in the performance of the system.

\section{Tests}
\label{sec:tests}
The testing phase is instrumental in affirming the integrity and functionality of the \ac{dsl} codebase developed for this thesis.
Tests serve as a critical mechanism for verifying the behaviour of the DSL across various scenarios, detecting potential bugs,
and validating adherence to specifications.

Since the \ac{dsl} is designed to be multiplatform, tests are written to ensure that the codebase is compatible with
different platforms, and the results are compared to ensure that the behaviour is consistent across platforms.


\section{Alchemist Simulations}
\label{sec:alchemist-simulations}


%examples
\section{Performance / Comparison}
\label{sec:performance-/-comparison}
To have a clear understanding of the performances of the new implementation, it is necessary to compare it with the other
\emph{ScaFi} and \emph{Protelis}' incarnations.
The comparison is made by running the same simulations on each incarnation and comparing the results.
The simulations are run on the same machine to ensure that the comparison is fair and that the differences are due to the
implementation and not to the hardware.

The performance of the new language has been evaluated through the implementation of 5 types of tests:
\begin{enumerate}
    \item Simple \textbf{state change} of the device, to represent the variation over time of the field;
    \item A \textbf{counter of the neighbours}, to represent the variation of the space;
    \item A \textbf{gradient}, which is a particular case of space-time variation, in which the value of a node is a
        function of the distance from a node considered as a source;
    \item A \textbf{channel with obstacles}, to represent the presence of obstacles in space that influence the communication between nodes;
    \item Simple \textbf{branching} operations, as it has been noticed that branching could be one of the most expensive operations in terms of execution time.
\end{enumerate}

Each test has been run with the same parameters in the three different incarnations with three different simulated times
inside the \emph{Alchemist} simulator, and the results have been analysed and will be further discussed in this section.

\paragraph{Machine Specifications}
The tests have been run on a machine with the following specifications:
%todo

\paragraph{Foreword}
During the evolution of this thesis, it has been noticed that the current implementation of the alignment in \emph{Collektive} may not be the optimal,
and therefore the results could be influenced by this.
A faster way to implement the alignment has been found, and it will be implemented in the future to have a more accurate comparison.

\paragraph{Field Evolution}
The first test has been implemented using the \texttt{repeat} construct.
It doesn't need any particular setup, as it is a simple state change of the device.

From the results, it is possible to see that the \emph{Protelis} implementation is faster than the \emph{Collektive} one,
which in turn is faster than the \emph{ScaFi} implementation.
%todo

\paragraph{Neighbour Counter}

\paragraph{Gradient}

\paragraph{Channel with Obstacles}

\paragraph{Branching}

\section{Discussion}

