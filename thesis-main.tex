\documentclass[12pt,a4paper,openright,twoside]{book}
\usepackage[utf8]{inputenc}
\usepackage{disi-thesis}
\usepackage{code-lstlistings}
\usepackage{notes}
\usepackage{shortcuts}
\usepackage{acronym}
\usepackage[inkscapeformat=png]{svg}

\newcommand{\fonte}[1]{{\color{gray} \small \hypersetup{citecolor=gray} Source: #1}}

\school{\unibo}
\programme{Corso di Laurea Magistrale in Ingegneria e Scienze Informatiche}
\title{A Kotlin multi-platform implementation of aggregate computing based on XC}
\author{Angela Cortecchia}
\date{\today}
\subject{Laboratio di Sistemi Software}
\supervisor{Prof. Pianini Danilo}
\cosupervisor{Dott. Farabegoli Nicolas}
\session{Quarta}
\academicyear{2022-2023}

%!TeX root = thesis-main.tex
\usepackage{color}

\definecolor{darkgreen}{rgb}{0,0.5,0}
\lstdefinelanguage{kt}{
    frame=single,
    basewidth=0.5em,
    language={kotlin},
    keywordstyle=\color{blue}\textbf,
    commentstyle=\color{darkgreen},
    keywordstyle=[2]\color{cyan},
    keywords=[2]{Aggregate, Field,InboundMessage, OutboundMessage, SingleOutboundMessage,Message,Path,YieldingScope,Network,State,
    AggregateContext,AggregateResult,YieldingContext,YieldingResult},
    keywordstyle=[3]\color{orange},
    keywords=[3]{exchange,exchanging,repeat,repeating,share,sharing,neighboringViaExchange,Collektive,aggregate,yielding},
    keywordstyle=[4]\color{teal}\textbf,
    keywords=[4]{ID,Scalar,Initial,Return,Payload,*,R,X,Init,Ret,Int,Double,Boolean},
    keywordstyle=[5]\color{olive},
    keywords=[5]{read,mapTo,stateAt,messagesAt,newField,excludeSelf,currentPath,mapToConstantField,getPosition,map,distanceTo,hood},
    keywordstyle=[6]\color{blue},
    keywords=[6]{also,it,run,let,context},
    keywordstyle=[7]\color{purple},
    keywords=[7]{neighborCounter,gradient,gradientEntrypoint,distances,distanceBetween,LocalSensing,DistanceSensor,sense,channel,channelWithObstacles},
    keywordstyle=[8]\color{violet},
    keywords=[8]{POSITIVE_INFINITY}
}
\acrodef{iot}[IoT]{Internet of Thing}
\acrodef{dsl}[DSL]{Domain Specific Language}
\acrodef{vm}[VM]{Virtual Machine}
\acrodef{ac}[AC]{Aggregate Computing}
\acrodef{fc}[FC]{Field Calculus}
\acrodef{cps}[CPS]{Cyber-Physical Systems}
\acrodef{cas}[CAS]{Collective Adaptive Systems}
\acrodef{jvm}[JVM]{Java Virtual Machine}
\acrodef{dsl}[DSL]{Domain Specific Language}
\acrodef{ast}[AST]{Abstract Syntax Tree}
\acrodef{ir}[IR]{Intermediate Representation}
\acrodef{ci}[CI]{Continuous Integration}
\acrodef{cd}[CD]{Continuous Deployment}
\acrodef{js}[JS]{JavaScript}
\acrodef{cicd}[CI/CD]{\ac{ci} and \ac{cd}}
\acrodef{kmp}[KMP]{Kotlin Multiplatform}

\mainlinespacing{1.241} % line spacing in mainmatter, comment to default (1)
\begin{document}

\frontmatter\frontispiece

\begin{abstract}
    The integration of technology in everyday activities is increasing, with objects being increasingly equipped with
    computational capabilities and interconnected to form the Internet of Things.
    Within these systems, different software components, and not necessarily only one per device, interact with each other
    to build innovative cyber-physical services (i.e., capable of creating a bridge between the real and virtual world).

    The solution that has the most success at the moment involves the use of Cloud Computing.
    This, although it provides virtualized resources on a large scale, has limits in terms of latency (due to the physical
    distance between machines) and scalability (due to centralization): for this reason, there is the aim to bring resources
    closer to the outer edge of the network, where, however, problems of interoperability between devices and complexity
    in the management of distributed resources increase.
    The design of applications capable of operating indistinctly on the cloud, on the edge, or even on a mesh network
    (in fact, on a computational continuum that goes from the edge to the cloud) can benefit from unconventional approaches,
    such as Aggregate Computing.

    In this context, the \emph{Collektive} project is proposed as a Domain-Specific Language for programming aggregate computing
    systems, based on Kotlin, a modern and multi-platform programming language.
    The main goal of this thesis is to enhance the \emph{Collektive} DSL, leveraging the principles of XC to improve the performance of the DSL,
    developing a multi-platform implementation of the DSL with kotlin-multiplatform, to allow the execution of the aggregate programs on different platforms.
    This enhancement was pursued to ensure its competitiveness against existing solutions and to enhance its usability.


\end{abstract}

\begin{dedication} % this is optional
To my family, a constant source of support,\\
who has always allowed me to choose\\ my own path with freedom and trust.
\end{dedication}

%----------------------------------------------------------------------------------------
\tableofcontents   
\listoffigures     % (optional) comment if empty
\lstlistoflistings % (optional) comment if empty
%----------------------------------------------------------------------------------------

\mainmatter

%! Author = angela
%! Date = 24/01/24
% !TeX root = ../thesis-main.tex

%----------------------------------------------------------------------------------------
\chapter{Introduction}
\label{ch:introduction}
%----------------------------------------------------------------------------------------
\section{Context}
\label{sec:context}

Computing devices are becoming cheaper and more \emph{ubiquitous}, increasing the complexity of distributed systems.
It is now common for individuals to own multiple computing devices of varying types,
resulting in technology becoming more integrated into daily activities.
Hence, the need to engineer and coordinate the operations in such systems, one way is to program and operate in terms of
    \empth{aggregates} devices, rather than manage each single device.
In actual fact, the coordination of macroscopic behavior in collective systems through a single program is a form of
~\empth{macroprogramming}.
However, this approach presents some primary challenges such as ensuring resilience, efficiency and privacy.

\paragraph{Macroprogramming}
%Macroprogramming: Concepts, State of the Art, and Opportunities of Macroscopic Behaviour Modelling

\paragraph{Self-Organisation}

\subsection{Aggregate Computing}
\label{subsec:aggregate-computing}

%TODO cite Modelling and simulation of Opportunistic IoT Services with Aggregate Computing and
% From distributed coordination to field calculus and aggregate computing

~\ac{ac} is a paradigm and engineering approach for the compositional development of self-adaptive
~\ac{iot} services from a global perspective.
It has been developed with the core idea of functionality
composing collective behaviours to achieve effective and resilient
complex behaviours in dynamic networks.
It views a given environment as a whole programmable entity whose parts collaboratively produce and consume services
across space and time.
~\ac{ac} is based on the principles of \empth{Field Calculus} that is a functional programming model used to specify
and compose collective behaviours with formally equivalent local and aggregate semantics.
The concept of \empth{Computational Fields} can be viewed as a distributed data structure,
with the aim of conceptually mapping each device to a value produced in a program, considering both
space and time.
Therefore, its structure supports the specification, analysis, simulation and runtime execution of \empth{collective}
or \empth{aggregate} services, independently of the specific~\ac{iot} architecture.

This paradigm has three key traits that characterize it:
    i)
Global stance with global-to-local mapping}, where the target of system design is the entire distributed
        \ac{iot} ecosystem,
    ii)
Behaviour compositionality}, whereby a rich collective service can be described in terms of the functional
        composition of simpler collective services, and
    iii)
Abstraction}, by which aggregate services enable adaptivity at different levels by abstracting from low-level
        issues and details.

%These characteristics play an essential role from the design perspective, where %

The main goal of ~\ac{ac} is to evaluate a single program that runs on multiple devices of different nature,

\paragraph{Alignment}

\subsection{Field Calculus}
\label{subsec:field-calculus}

\subsection{XC}
\label{subsec:xc}


\section{Motivations}
\label{sec:motivations}

\subsection{Heterogeneity limitations}
\label{subsec:heterogeneity-limitations}

\subsection{Goal}
\label{subsec:goal}


\section{State of Art}
\label{sec:state-of-art}

\subsection{Protelis}
\label{subsec:protelis}

\subsection{ScaFi}
\label{subsec:scafi}

\subsection{FCPP}
\label{subsec:fcpp}


%
%Write your intro here.
%\sidenote{Add sidenotes in this way. They are named after the author of the thesis}
%
%You can use acronyms that your defined previously,
%such as \ac{IoT}.
%%
%If you use acronyms twice,
%they will be written in full only once
%(indeed, you can mention the \ac{IoT} now without it being fully explained).
%%
%In some cases, you may need a plural form of the acronym.
%%
%For instance,
%that you are discussing \acp{vm},
%you may need both \ac{vm} and \acp{vm}.
%
%\paragraph{Structure of the Thesis}
%
%\note{At the end, describe the structure of the paper}
%

%! Author = angela
%! Date = 24/01/24
% !TeX root = ../thesis-main.tex

\chapter{Contributions}
\label{ch:contributions}
In this chapter, will be presented the main contributions of this thesis.

\section{Collektive}
\label{sec:collektive}

\emph{Collektive} is a framework designed to simplify the definition of \ac{ac} systems.

The main objective of this technology is to facilitate the development of aggregate programs that can be executed on a
variety of computing systems, such as mobile and wearable devices, computers, and the cloud.
This allows for interoperability and communication between these systems, despite their different nature.

To achieve this, \emph{Collektive} uses the \ac{fc} model to provide a straightforward and intuitive method for defining
an aggregate program, without the need for low-level coding.

In addition, \emph{Collektive} has been developed to be multiplatform, so it can be executed on different systems thanks
to the use of \emph{Kotlin Multiplatform}.

As for the feature solution of alignment for the correct functioning of aggregate programming,
it has been developed a compiler plugin with the purpose of annotating the functions that are aligned;
those paths will be used for the actual alignment of the nodes.

\paragraph{Project Structure}
The project is subdivided into different submodules (as in \Cref{fig:pacakges}), each with a specific purpose:

\begin{enumerate}
    \item \textbf{alchemist-incarnation-collektive}: contains the pieces for the Alchemist integration,
        in order to run the aggregate programs created with Collektive on the simulator.
    \item \textbf{dsl}: is the core of the project, contains the actual implementation of the logic and the \ac{ac} operators
        and relative tests.
    \item \textbf{plugin}: subdivided into:
        \begin{enumerate}
            \item \emph{gradle-plugin}: used by gradle project in order to use the \emph{compiler-plugin}.
            \item \emph{compiler-plugin}: used to keep tracks of the stack at runtime, foreach aggregate program.
        \end{enumerate}
    \item \textbf{buildSrc}: contains the main configuration for running the multiplatform project.
\end{enumerate}

\begin{figure}[h!]
    \centering
    \includegraphics[width=0.7\textwidth]{figures/packages}
    \caption{Packages diagram of the Collektive project.}
    \label{fig:pacakges}
\end{figure}

Regarding the examples, there is a specific repository called \textbf{collektive-examples} that contains some samples of
aggregate programs to show how to use the \emph{Collektive} framework.

\section{DSL}
\label{sec:dsl}

In this thesis, the original implementation of the \ac{dsl} of \emph{Collektive} will be modified to allow the use of \xc{}
and to improve its performance.

\paragraph{Structure}
%todo add class diagrams

\subsection{XC in Collektive}
\label{subsec:exchange-in-collektive}

Thanks to the design of \xc{}, it is possible to implement the methods proposed by \emph{field calculus}
(~\ref{par:syntax-of-field-calculus}) in terms of \emph{exchange} (~\ref{par:communication-in-xc}).

The syntax of \emph{XC} allows for sending messages to specific nodes, enabling the implementation of \emph{field calculus}
operations through message exchange.

The \emph{exchange} communication is based on \emph{anisotropic} communications, meaning that it has not the same properties
or characteristics in all directions (\Cref{fig:anisotropic}); therefore, messages have custom values sent to different neighbours.

\begin{figure}[h!]
    \centering
    \includegraphics[width=0.2\textwidth]{figures/anisotropic}
    \caption{Anisotropic communication.}
    \label{fig:anisotropic}
\end{figure}

This concept can be extended to the \texttt{share} function of field calculus, with the difference that the
operation of \texttt{share} is based on \emph{isotropic} communication, meaning that properties are uniform in all directions
(\Cref{fig:isotropic}); therefore, messages have the same value sent to all neighbours.

\begin{figure}[h!]
    \centering
    \includegraphics[width=0.2\textwidth]{figures/isotropic}
    \caption{Isotropic communication.}
    \label{fig:isotropic}
\end{figure}

\paragraph{Syntax}

All the \ac{dsl} has been modified to use \texttt{exchange} for the implementation of the other constructs such as \texttt{share}
and \texttt{nbr}, which is called \texttt{neighboring}.
Only the \texttt{rep} construct has not been implemented in terms of \texttt{exchange}, as it is a function that allows iterating
over oneself, it's better for neighbours not to receive messages of any kind, also for security and privacy reasons.

As the original implementation, it supports the evaluation of \emph{fields}.

As seen in \ref{par:communication-in-xc}, the \texttt{exchange} function can send and return the same result, or it
can send a message and return a different result; both cases have been implemented.

%todo add code snippet of exchange and exchanging

\subsection{Messages}
\label{subsec:messages}
%message modelling (in scafi there is an action that computes the messages and then creates a reaction that
%sends the message) we do it in a different way

\subsection{Syntax}
\label{subsec:syntax}
%code
%rep non serve farla con xc perche non vogliamo che gli altri nodi ricevano il messaggio, cose che non gli interessano

\section{Plugin Extensions}
\label{sec:plugin-extensions}

\paragraph{Alignment}
%how

\section{Incarnation}
\label{sec:incarnation}
%how, why

%gradle task

\section{Technologies}
\label{sec:technologies}
%multiplatform

\section{Implementation}
\label{sec:implementation}
%diagrams




%! Author = angela
%! Date = 24/01/24
% !TeX root = ../thesis-main.tex

\chapter{Validation}
\label{ch:validation}

\section{Tests}

\section{Alchemist Simulations}
\label{sec:alchemist-simulations}
%examples
\section{Performance / Comparison}
\label{sec:performance-/-comparison}



%You may also put some code snippet (which is NOT float by default), eg: \cref{lst:random-code}.
%
%\lstinputlisting[float,language=Java,label={lst:random-code}]{listings/HelloWorld.java}
%
%\section{Fancy formulas here}


%! Author = angela
%! Date = 24/01/24
% !TeX root = ../thesis-main.tex

\chapter{Conclusions}
\label{ch:conclusions}

The primary objective of this thesis was to enhance the current \ac{dsl} for aggregate computing \emph{Collektive},
leveraging the principles of ~\xc~ to enhance the performance of the \ac{dsl}.
The other main goal was to develop a multiplatform implementation of the \ac{dsl} with \ac{kmp}, to allow the execution
of the aggregate programs on different platforms.

This enhancement was pursued to ensure its competitiveness against existing solutions and to enhance its usability.
Additionally, there was as a goal to develop a corresponding incarnation for conducting simulations within the \emph{Alchemist} environment.

A key aspect in the development of the \ac{dsl} was to ensure a good and functional alignment, to guarantee
the correct functioning of the language and its features, maintaining a good result in terms of performance.
The communication between the nodes, intended as the exchange of messages, was also a fundamental aspect to guarantee
the correct functioning of the language and its execution speed.
All this was possible thanks to the implementation of an expressive and flexible language, which made it easy to define complex systems.

\section{Future Works}
\label{sec:future-works}
In the near future, the development of the \ac{dsl} will continue, as it has been possible to win a research grant from the GARR organization.
It will therefore be possible to further deepen the research and develop new features for the language.

As future works have been identified the followings:
\begin{itemize}
    \item \textbf{Further optimisations of the \ac{dsl}}: the \ac{dsl} will be further optimised to improve its performance;
    \item \textbf{Alignment optimisation}: to improve performances and security looking forward to the execution of the aggregate
        program on different platforms and environments;
    \item \textbf{Development of a standard library}: to provide modules and functionalities to simplify the writing of the aggregate program.
        such as self-stabilizing functions that could encompass a range of strategies commonly employed to attain adaptable
        and resilient decentralised behaviours.
        These functions aim to conceal complexities by leveraging constructs from field calculus;
    \item \textbf{Creation of demonstrations}: to illustrate the possibility of running the same aggregate program on
        different platforms simultaneously, including server JVM, Android devices, iOS, and web browsers.
\end{itemize}

%----------------------------------------------------------------------------------------
% BIBLIOGRAPHY
%----------------------------------------------------------------------------------------

\backmatter

\nocite{*} % comment this to only show the referenced entries from the .bib file

\bibliographystyle{alpha}
\bibliography{bibliography}


\begin{acknowledgements} % this is optional
    Optional. Max 1 page.
\end{acknowledgements}

\end{document}